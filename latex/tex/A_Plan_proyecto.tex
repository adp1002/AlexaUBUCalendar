\apendice{Plan de Proyecto Software}

\section{Introducción}

En este apartado se va a comentar la planificación temporal del proyecto así como la metodología que se ha seguido para su desarrollo y la viabilidad del proyecto en el marco económico y legal.

\section{Planificación temporal}

En el desarrollo se ha empleado SCRUM, que es un marco de trabajo para desarrollo ágil de software, pero en una versión adaptada a un equipo de 1 persona, a diferencia de los equipos normales que suelen ser de 5 a 9 personas. También se ha cambiado la frecuencia de las reuniones de seguimiento, siendo semanales en vez de diarias.

Estas reuniones se han utilizado para exponer el trabajo realizado, los problemas que se han encontrado durante su realización, mostrar el avance del proyecto y fijar nuevas tareas para continuar con el proyecto.

Como es normal utilizando metodología ágil el desarrollo ha sido iterativo organizado por \textit{sprints}. La duración de los \textit{sprints} ha sido de una, dos y tres semanas.

\subsection{Sprint 1}

El primer \textit{sprint} duró tres semanas (del 17 de febrero al 2 de marzo). La primera reunión de seguimiento se centró en explicar detalladamente los objetivos del trabajo. Gran parte del trabajo realizado en este \textit{sprint} se centró en investigar y aprender como funcionan los asistentes virtuales, en concreto Alexa, así como centrarme en leer trabajos relacionados con lo que se pretendía en el proyecto para analizar que cosas están bien hechas, que se puede cambiar y la viabilidad de algunas otras cosas.

También aprendí cómo funciona Moodle, que son los \textit{web services} y como utilizarlos. Finalmente, intentando hacer el prototipado de una primera versión de \textit{Skill} de Alexa, se decidió cambiar el asistente con el que se iba a desarrollar el trabajo.

\subsection{Sprint 2}

Este \textit{sprint} tuvo una duración de tres semanas (del 2 de marzo al 12 de marzo). En este \textit{sprint} me dediqué a plantear si seguir con este proyecto, cambiar totalmente o hacer alguna modificación. Finalmente me decanté por continuar con el asistente de voz pero necesitaba encontrar un asistente de voz que no me fuera a dar problemas. Se contempló el uso de Google Assistant pero no estaba seguro de que no me fuera a dar problemas similares que los que tuve con Alexa.

Continuando la búsqueda de un asistente me encontré con Mycroft, que era open source y no tenía ningún tipo de limitación, además de estar bastante extendido para ser un asistente de este tipo. Así que decidí usar Mycroft para continuar el proyecto.
Tras comentarle la decisión al tutor, empleé el resto del \textit{sprint} para prototipar una primera version de la \textit{Skill} y contemplar nuevas opciones que se habían abierto al usar Mycroft en vez de Alexa, como un cliente gráfico.

\subsection{Sprint 3}

El tercer \textit{sprint} duró una semana (del 18 al 25 de marzo) y se desarrolló una versión inicial de la interfaz gráfico. También se implementó el acceso al calendario mediante los \textit{web services} de Moodle. Finalmente estuve haciendo tests y investigando si podía haber algún tipo de problema en cuanto al número de accesos.

\section{Estudio de viabilidad}

\subsection{Viabilidad económica}

\subsection{Viabilidad legal}


