\capitulo{4}{Técnicas y herramientas}

\section{Metodología de desarrollo}

Para el desarrollo del proyecto se ha utilizado metodología ágil, en concreto SCRUM. Se ha adoptado una estrategia de desarrollo iterativa, con sprints bimensuales en los que se han realizado las entregas parciales. Las reuniones de estado del proyecto se han realizado semanalmente, a diferencia de lo normal en SCRUM que se realizan diariamente.

\section{Moodle}

\href{https://moodle.org/}{Moodle} es una plataforma de aprendizaje que permite a docentes y alumnos impartir y recibir clases a distancia, así como facilitar la gestión de los cursos. Es una herramienta de código abierto y por lo tanto cualquiera puede utilizarlo. En su página se pueden encontrar versiones de prueba como \href{https://school.moodledemo.net/}{Mount Orange School}

\section{PyQt5}

\href{https://doc.qt.io/qtforpython/}{PyQt5} es un binding (una adaptación de una biblioteca para que sea usada en otro lenguaje). Qt es un framework multiplataforma orientado a objetos para desarrollar programas que utilizan interfaces gráficas de usuario. Es software libre y de código abierto.

\section{requests}

\href{https://requests.readthedocs.io/es/latest/}{Requests} es una biblioteca para Python que permite enviar peticiones HTTP de forma muy sencilla.

\section{MiKTeX}


\section{TeXstudio}

\section{Python}

\section{Java}

