\capitulo{6}{Trabajos relacionados}

El campo de este proyecto está muy inexplorado, aunque si que hay algún trabajo bastante similar a este proyecto.

\textbf{ALEXA SKILL VOICE INTERFACE FOR THE MOODLE LEARNING MANAGEMENT SYSTEM}\cite{meltonALEXASKILLVOICE} es un proyecto de una \textit{skill} para Alexa que permite a los usuarios acceder a anuncios de un curso, notas y eventos próximos.

Sin embargo, en el campo de los asistentes de voz existen múltiples proyectos en gran variedad de campos. En la \href{https://market.mycroft.ai/skills}{página} de Mycroft para conseguir nuevas \textit{skills} puedes por ejemplo \href{https://market.mycroft.ai/skills/3422768c-8f3a-45c4-ad47-a86b9eba6a02}{controlar las luces}, controlar \href{https://market.mycroft.ai/skills/b213f51b-c8aa-455e-80ec-f2489cf20eca}{las horas de buses} o \href{https://market.mycroft.ai/skills/5cff8320-4738-4e06-ac87-a5c3e6af5c25}{jugar a un juego}.

Además, el uso de los asistentes de voz se puede escalar a proyectos mucho mayores que los anteriores. Por ejemplo, la universidad de Saint Louis instaló 2300 \textit{Echo Dot}\cite{SLUAlexaProject}, que son los dispositivos para el asistente de voz Alexa, y crearon una \href{https://www.amazon.com/Saint-Louis-University-Ask/dp/B07YDNW9RN}{\textit{skill} personalizada} que ofrecía muchas posibilidades para los estudiantes, como preguntar por eventos y organizaciones, hora de cierre de la biblioteca o dónde se puede comer.

Las ventajas del proyecto de la universidad de Saint Louis respecto al mío son la facilidad de uso, ya que los estudiantes solo tienen que aprender a utilizar la \textit{skill} para tener acceso a la información que esta les provee, y que, siendo un proyecto mucho más grande hecho por la universidad, los usuarios tienen acceso a más funcionalidad, por ejemplo saber las cafeterías disponibles para comer. La desventaja principal sería el coste del proyecto, la instalación de los 2300 \textit{Echo Dot} rondaría los 80.500€ si cada dispositivo les cuesta 35€, que es lo que valen en la página de Amazon. Solo eso costaría 16 veces más que este proyecto.

Por último, un proyecto hecho en la universidad de Burgos, UBUAssistant\cite{DanielSantidrianUBUassistant}, es un asistente virtual que permite al usuario navegar por la página de la universidad de Burgos realizando preguntas por texto y el asistente devolverá la respuesta si existe o una sugerencia para localizar esta respuesta. La ventaja de utilizar este asistente es que el usuario no tiene que navegar por distintos menús y páginas, solo necesita saber que es lo que quiere encontrar y preguntárselo al asistente.