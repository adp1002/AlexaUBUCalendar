\apendice{Especificación de diseño}

\section{Introducción}

En este apartado se va a explicar como está diseñado el proyecto.

\section{Diseño de datos}

Los datos están correlacionados con el formato con el que se guardan en Moodle, aunque solo están modelados aquellos datos que me son útiles. En el paquete /src/model/ están las clases que modelan estos datos y son:
\begin{itemize}
	\item \textbf{User} para guardar datos relativos al usuario.
	\item \textbf{Course} para guardar los datos de los cursos.
	\item \textbf{Forum} para los foros de un curso.
	\item \textbf{Discussion} para las discusiones de un foro.
	\item \textbf{GradeItem} para las notas.
	\item \textbf{Event} para los eventos del calendario.
\end{itemize}

\tablaSmallSinColores{Diccionario de datos de User}{l c c}{data_user}
	{ \multicolumn{1}{l}{Atributo} & Tipo de dato & Descripción \\}{
		user\_id     & int      & Identificador del usuario   \\
		courses    & dict    & Diccionario que contiene los cursos del usuario\\
						   & & La clave es la id del curso y\\
						   & & el valor es el objeto Course asociado \\
	}

\tablaSmallSinColores{Diccionario de datos de Course}{l c c}{data_course}
	{ \multicolumn{1}{l}{Atributo} & Tipo de dato & Descripción \\}{
		course\_id     & String      & Identificador del curso   \\
		name	& String	& Nombre del curso	\\
		grades    & Array    & Vector que contiene las calificaciones\\
						   & &  del usuario de ese curso \\
		events    & Array    & Vector que contiene los eventos del curso \\
		forums    & Array    & Vector que contiene los foros de ese curso \\
	}

\tablaSmallSinColores{Diccionario de datos de Forum}{l c c}{data_forum}
	{ \multicolumn{1}{l}{Atributo} & Tipo de dato & Descripción \\}{
		id     & int      & Identificador del foro   \\
		name    & String    & Nombre del foro \\
		discussions    & Array    & Vector que contiene las discusiones del curso \\
	}

\tablaSmallSinColores{Diccionario de datos de Discussion}{l c c}{data_discussion}
	{ \multicolumn{1}{l}{Atributo} & Tipo de dato & Descripción \\}{
		discussion\_id     & int      & Identificador del foro \\
		name    & String    & Nombre de la discusión \\
	}

\tablaSmallSinColores{Diccionario de datos de GradeItem}{l c c}{data_gradeitem}
	{ \multicolumn{1}{l}{Atributo} & Tipo de dato & Descripción \\}{
		value     & int      & Valor de la calificación   \\
		name    & String    & Nombre del \textit{item} \\
		type	& String	& Tipo del \textit{item} \\
	}

\tablaSmallSinColores{Diccionario de datos de Event}{l c c}{data_event}
	{ \multicolumn{1}{l}{Atributo} & Tipo de dato & Descripción \\}{
		name    & String    & Nombre del evento \\
		date	& String	& Fecha de finalización del evento \\
	}
	

\imagen{diagrama_clases}{Diagrama de clases}
\section{Diseño procedimental}

A continuación se van a detallar las conexiones realizadas por la aplicación para realizar la conexión con Moodle y conseguir el token del usuario y de los servicios de Mycroft.

Lo primero que se hace es enviar una petición usando como parámetros en la URL el usuario y la contraseña y como respuesta se obtiene el token del usuario, necesario para realizar el resto de peticiones de los \textit{web services}

Después se obtiene información relevante para la aplicación a través del \textit{web service} \textbf{core\_webservice\_get\_site\_info}. La respuesta de esta petición devuelve el idioma de la plataforma de Moodle y la id del usuario de Moodle, que será necesaria para utilizar otros \textit{web services}

Por último se obtienen los cursos en los que está inscrito el usuario mediante el \textit{web service} \textbf{core\_enrol\_get\_users\_courses}
\imagen{diagrama_secuencia}{Diagrama de secuencia de la conexión con Moodle}

Si el usuario interactúa con la aplicación mediante voz el \textbf{MessageBus} manda el archivo mp3 con el comando al componente \textbf{Voice}. Después éste utiliza el motor de STT para convertir el archivo mp3 a una cadena de texto con la pregunta, la \textit{utterance}.
La \textit{utterance} vuelve al \textbf{MessageBus}, quien la envía al componente \textbf{Skills}. El componente \textbf{Skills} utilizará el controlador de \textit{intents} correspondiente para obtener el resultado.
El resultado va de vuelta al \textbf{MessageBus} hacia el componente \textbf{Voice} de nuevo, que mediante el motor de TTS lo convertirá a un archivo mp3 y lo enviará al \textbf{MessageBus}, donde se reproducirá.

Si el usuario interactúa con la aplicación vía texto el proceso es el mismo pero omitiendo el paso del motor de STT.

\imagen{voice_command_sequence}{Diagrama de secuencia de la interacción del usuario por voz}
\imagen{text_input_sequence}{Diagrama de secuencia de la interacción del usuario por texto}

\section{Diseño arquitectónico}

El proyecto realmente consiste en dos aplicaciones, una la aplicación con la interfaz gráfica y la otra Mycroft. Ambas aplicaciones utilizan una arquitectura cliente-servidor, que es una arquitectura en la que el cliente realiza peticiones a través de un medio y el servidor envía una respuesta en función de la petición a través del mismo medio. La aplicación gráfica actúa como cliente y el servidor es la API REST de la plataforma de Moodle, es decir, los \textit{web services}. Mycroft tiene como cliente sus \textit{skills} que se comunican con dos servidores, uno para las funciones de STT y otro para TTS. Y entre las dos aplicaciones también existe una arquitectura cliente-servidor, siendo la aplicación gráfica el servidor y los clientes las diferentes \textit{skills}.
\imagen{diagrama_despliegue}{Diagrama de despliegue}