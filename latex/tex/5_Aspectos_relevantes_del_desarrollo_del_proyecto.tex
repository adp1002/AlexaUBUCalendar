\capitulo{5}{Aspectos relevantes del desarrollo del proyecto}

Este apartado pretende comentar los aspectos importantes y problemas que han surgido con la realización del proyecto, así como las decisiones de añadir o no más funcionalidad a la aplciación.
\section{Utilización de AmazonWebServices y Alexa para la realización del proyecto}
La idea inicial era utilizar los servicios web de Amazon (Amazon Lambda) y la consola de desarrollador (Amazon Developer) para crear la Skill de Alexa y hostearla en sus servidores.

Ya que nunca había usado nunca estos servicios ni había desarrollado ninguna Skill o aplicación asistente de voz similar, leyendo la documentación de Alexa encontré una guía de cómo crear tu primera skill.
\subsection{Amazon Web Services}
Los Amazon Web Services, AWS a partir de ahora, es un conjunto de servicios de computación en la nube. En concreto, de todos estos servicios se utilizó Amazon Lambda, que es una plataforma sin servidor basada en eventos, que es la base de los AWS. Lambda ejecuta código como respuesta a eventos y gestiona los recursos necesitados por el código ejecutado.
\subsection{Amazon Alexa}
Es un servicio de voz en la nube para dispositivos de Amazon y dispositivos de terceros que usan Alexa.
\section{Problemas con Alexa}
\section{Mycroft como alternativa}