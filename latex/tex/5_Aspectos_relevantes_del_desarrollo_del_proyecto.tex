\capitulo{5}{Aspectos relevantes del desarrollo del proyecto}

Este apartado pretende comentar los aspectos importantes y problemas que han surgido con la realización del proyecto, así como las decisiones de añadir o no más funcionalidad a la aplicación.

\section{Utilización de AmazonWebServices y Alexa para la realización del proyecto}

La idea inicial era utilizar los servicios web de Amazon (Amazon Lambda) y la consola de desarrollador (Amazon Developer) para crear la \textit{Skill} de Alexa y alojarla en sus servidores.

Ya que nunca había usado estos servicios ni había desarrollado ninguna \textit{Skill} o aplicación asistente de voz similar, leyendo la documentación de Alexa encontré una guía de cómo crear tu primera \textit{skill} \cite{DevelopingYourFirst}. En un principio solo utilicé la guía para configurar la \textit{Skill} dentro de Amazon Lambda y Amazon Developer, usando el código que había hecho ya utilizando los webservices de Moodle.

\subsection{Problemas con Alexa}

Sin embargo, cuando fui a hacer la prueba de esta primera versión de la \textit{Skill}, no conseguía ninguna respuesta. Así que, al ser la primera vez que usaba la libreria de Amazon \textit{Skills} Kit y la primera vez también que hacía un programa de este tipo, utilicé el código de prueba básico, que es un \textit{hello world} para la \textit{Skill}, que viene en la guía que estaba siguiendo. Y, a pesar de seguir todos los pasos, intentándolo varias veces por si me había saltado algo, seguía sin recibir ninguna respuesta.

En uno de esos reintentos por conseguir que funcionara, me di cuenta de que salía un aviso de que el archivo con el proyecto de la \textit{Skill} era demasiado pesado (superaba los 10MB). Aún no tengo claro que fuera ese el problema, ya que ese proyecto tenía lo mínimo necesario para hacer una \textit{Skill}, pero de todas formas decidí continuar con el proyecto por otro camino.

\section{Mycroft como alternativa}

Esta herramienta es de código abierto, por lo que tuve la posibilidad de ver cómo están implementados los diferentes servicios del asistente de voz, modificarlos, etc. Además, para crear una \textit{Skill} de Alexa, estás restringido en cierto modo qué puedes y qué no puedes hacer, problema que no existe con Mycroft que te da libertad total para crear lo que quieras. Tampoco tiene las restricciones de tamaño que me impedían continuar con el proyecto en Alexa, así que por todas estas características decidí volver a empezar pero usando Mycroft esta vez.

Pero con Mycroft no son todo ventajas. De momento esta aplicación no se puede instalar en Windows, teniendo que usar una máquina virtual de Ubuntu o distribución de Linux para usarla, aunque se puede usar en otras plataformas como Android, Raspberry Pi y Docker, de lo que se hablará en el capítulo 7, líneas de trabajo futuras. Al ser una aplicación cliente el usuario necesita instalarla para usar la \textit{Skill}.

\section{Aplicación cliente}

En el anterior apartado he comentado la libertad que te ofrece Mycroft. Pues bien, esta herramienta se ejecuta como una aplicación cliente, a diferencia de Alexa que es un servicio en la nube y que necesitas emplear los servicios de Amazon para usar la \textit{Skill}. Y, aunque Mycroft también depende de servicios, como el de text-to-speech o el de speech-to-text, y por ello una conexión a internet, se puede crear una aplicación cliente usando estos servicios ya que la propia aplicación de Mycroft es cliente.

Así que debido a esto el proyecto tomó un camino distinto al inicial, teniendo la posibilidad de crear una aplicación gráfica que resultaría en una aplicación más atractiva y completa en general.

\section{Comunicación entre procesos}

En el inicio de esta nueva etapa surgió un problema que no había planteado cuando decidí usar Mycroft como alternativa a Alexa. Al ejecutar la aplicación gráfica para loguearse en Moodle e invocar una \textit{Skill}, se crea un nuevo programa con esa \textit{Skill}, por lo que la comunicación entre la aplicación gráfica y \textit{Skill} se complicó. Como solución utilicé comunicación entre procesos mediante sockets, lo que resultó más sencillo de lo que en un primer momento me había imaginado.

\section{Webservices de Moodle}

Uno de los objetivos de este proyecto es que el usuario pueda consultar sus calificaciones, sin embargo solo existía un \textit{web service} que permitía obtener las calificaciones finales de los cursos del usuario. Entonces, a falta de un \textit{web service} que me permitiera obtener las calificaciones completas de cada curso del usuario, se optó por utilizar \textit{web scraping}.

Un tiempo después de haber implementado esta forma de obtener las calificaciones del usuario Moodle actualizó sus \textit{web services} y en esta actualización existía un \textit{web service}, \textbf{gradereport\_user\_get\_grade\_items} \cite{MDL64298GradereportUser}, que permitía obtener las calificaciones de un curso de un usuario, así que deseché la implementación usando \textit{web scraping} y lo implementé usando este \textit{web service} para continuar con la compatibilidad de la aplicación para cualquier Moodle.

\section{Deshabilitar módulos del asistente}

Otra adición importante en la aplicación es la posibilidad de activar/desactivar distintos \textit{intents}. Sin embargo, en el estado actual de Mycroft no es posible hacer cambios de este nivel de granularidad, pudiendo únicamente desactivar o activar \textit{Skills} completas. Así que la solución fue separar la \textit{Skill} en varias, en función de qué hacen, consiguiendo tres \textit{Skills}, una para los eventos del calendario, otra para distintas acciones de un curso y otra para las notas.

\section{Skill mycroft-volume.mycroftai}

De un día para otro empecé a tener problemas con el sonido en general de la aplicación. A veces el micrófono no estaba activo, otras veces se desactivaba tras detectar la \textit{wake word}, el audio del TTS se silenciaba, etc. Sin embargo no había realizado ningún cambio relacionado con el audio o Mycroft en sí, así que tras mucho probar, buscar en los foros de Mycroft y mirar los logs de la aplicación encontré un post que hablaba sobre problemas muy similares a los que estaba teniendo y que estaban relacionados con la \textit{skill} \textbf{mycroft-volume.mycroftai}, que es una \textit{skill} propia de Mycroft. Tras desactivarla para comprobar que ese era el problema todo empezó a funcionar como debería así que por esta razón decidí dejar esta \textit{skill} desactivada para la aplicación.