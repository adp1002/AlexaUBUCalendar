\capitulo{3}{Conceptos teóricos}

A continuación se va a detallar qué es un asistente de voz, una \textit{Skill}, qué elementos tiene y cómo funcionan. También se explicará el funcionamiento de \textit{Mycroft}, que es el asistente de voz que se ha usado para el desarrollo del proyecto.

\section{Asistente de voz}

Un asistente de voz es un programa que actúa para un usuario, al que ayuda automatizando y realizando tareas con una interacción hombre-máquina mínima. La interacción entre el asistente y la persona debe ser una interacción lo más natural posible, es decir, la persona se comunica con la voz como es habitual y el asistente procesa, interpreta y responde de la misma forma.

\section{Skill}

Una \textit{Skill} es una aplicación diseñada para un asistente de voz. Estas \textit{skills} tienen diferentes elementos que hacen que funcionen, como la \textit{wake word}, las \textit{utterances}, los \textit{intents} y los \textit{dialogs}.

\subsection{Wake Word}

La \textit{wake word} es una palabra o palabras que tiene que decir el usuario para que el asistente comience a escuchar. En el asistente de voz que se ha usado para el proyecto, la \textit{wake word} es configurable, pero por defecto es "Hey Mycroft".

\subsection{Utterance}

Una \textit{utterance} es la frase completa que dice el usuario a continuación de la \textit{wake word} y que sirve como inicio del proceso de búsqueda de la \textit{Skill} correspondiente a la frase dicha. Un ejemplo de \textit{utterance} sería: "¿qué temperatura hace en Burgos?"

\subsection{Intent}

Un \textit{intent} es la parte de la frase o \textit{utterance} que el asistente de voz detecta como la parte necesaria para la invocación de una \textit{Skill}. Cada \textit{Skill} tiene un \textit{intent} asociado. Siguiendo con el ejemplo anterior, el intent sería: temperatura Burgos

\subsection{Dialog}

Este término es particular de Mycroft. Un \textit{dialog} es una frase que dice el propio asistente, normalmente en respuesta a una pregunta por parte del usuario. Los \textit{dialog} son independientes de las \textit{Skill}, pero es normal que cada \textit{Skill} tenga un \textit{dialog}. Es el último paso cuando usamos una \textit{Skill}, pero no tiene por qué estar presente. Finalizando con el ejemplo, el \textit{dialog} sería: "Hay 11ºC en Burgos".