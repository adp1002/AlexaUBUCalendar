\capitulo{2}{Objetivos del proyecto}

Este apartado explica de forma precisa y concisa cuales son los objetivos que se persiguen con la realización del proyecto. Se puede distinguir entre los objetivos marcados por los requisitos del software a construir y los objetivos de carácter técnico que plantea a la hora de llevar a la práctica el proyecto.

\section{Objetivos generales}

\begin{itemize}
	\item Diseñar e implementar una aplicación que permita a los usuarios de una plataforma de Moodle interactuar con esta.
	\item Conseguir que la interacción con el asistente sea lo más natural posible.
	\item Programar la aplicación para que añadir futura funcionalidad o realizar cambios sea sencillo.
\end{itemize}

\section{Objetivos técnicos}

\begin{itemize}
	\item Aplicar los conocimientos adquiridos como patrones de diseño, buenas prácticas de programación, etc.
	\item Familiarizarme con el uso de una REST API, como por ejemplo los \textit{web services} de Moodle.
	\item Acostumbrarme a empezar un proyecto desde cero, tanto en el diseño de este como en su implementación.
	\item Utilizar las herramientas existentes en el asistente de voz
\end{itemize}