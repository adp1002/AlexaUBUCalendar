\apendice{Especificación de Requisitos}

\section{Introducción}

Este apartado va a explicar los objetivos del proyecto y los requisitos necesarios para cumplirlos.

\section{Objetivos generales}

Los objetivos del proyecto son la recogida de datos de una plataforma de Moodle a través de sus \textit{web services} y enviárselos al asistente de voz para que la interacción con la aplicación sea más interactiva. Todo esto a través de una interfaz gráfica que le permite al usuario configurar el asistente virtual en ejecución.

\section{Catalogo de requisitos}

\subsection{Requisitos funcionales}
\begin{itemize}
	\item RF-1 Guardar los credenciales de inicio de sesión: la aplicación debe permitir al usuario recordar el usuario y host para ejecuciones posteriores.
	\item RF-2 Obtener los datos de Moodle: la aplicación debe obtener distintos datos de Moodle.
	\begin{itemize}
		\item RF-2.1 Obtener calificaciones del usuario: la aplicación debe ser capaz de obtener las notas finales del usuario, así como de un curso concreto.
		\item RF-2.2 Obtener eventos del calendario: la aplicación debe ser capaz de obtener los eventos próximos, de un día concreto y de un curso concreto.
		\item RF-2.3 Obtener los foros de un curso: la aplicación debe ser capaz de obtener los foros, así como sus discusiones, de un curso concreto.
	\end{itemize}
	\item RF-3 Interacción por voz: la aplicación debe permitir al usuario interactuar con ella mediante comandos de voz
	\item RF-4 Interacción por texto: la aplicación debe permitir al usuario interactuar con ella mediante texto
	\item RF-5 Logs de ejecución: la aplicación debe guardar y visualizar mensajes describiendo la ejecución.
	\begin{itemize}
		\item RF-5.1 Guardar logs: la aplicación debe guardar logs de la ejecución.
		\item RF-5.1 Visualizar logs: la aplicación debe mostrar logs de la ejecución.
	\end{itemize}
	\item RF-6 Activar/desactivar \textit{skills}: la aplicación debe permitir al usuario activar o desactivar \textit{skills} en ejecución.
	\item RF-7 Control del micrófono: la aplicación debe permitir al usuario silenciar o activar el micrófono para utilizar los comandos de voz.
\end{itemize}

\subsection{Requisitos no funcionales}
\begin{itemize}
	\item RNF-1 Usabilidad: el usuario debe ser capaz de utilizar la aplicación sin mucho esfuerzo.
	\item RNF-2 Internacionalización: la aplicación debe permitir introducir nuevos idiomas y elegir entre estos.
\end{itemize}

\section{Especificación de requisitos}
\imagen{diagrama_casos_uso}{Diagrama de casos de uso}

