\apendice{Especificación de Requisitos}

\section{Introducción}

Este apartado va a explicar los objetivos del proyecto y los requisitos necesarios para cumplirlos.

\section{Objetivos generales}

Los objetivos del proyecto son la recogida de datos de una plataforma de Moodle a través de sus \textit{web services} y enviárselos al asistente de voz para que la interacción con la aplicación sea más interactiva. Todo esto a través de una interfaz gráfica que le permite al usuario configurar el asistente virtual en ejecución.

\section{Catalogo de requisitos}

\subsection{Requisitos funcionales}
\begin{itemize}
	\item RF-1 Guardar los credenciales de inicio de sesión: la aplicación debe permitir al usuario recordar el usuario y host para ejecuciones posteriores.
	\item RF-2 Obtener los datos de Moodle: la aplicación debe obtener distintos datos de Moodle.
	\begin{itemize}
		\item RF-2.1 Obtener calificaciones del usuario: la aplicación debe ser capaz de obtener las notas finales del usuario, así como de un curso concreto.
		\item RF-2.2 Obtener eventos del calendario: la aplicación debe ser capaz de obtener los eventos próximos, de un día concreto y de un curso concreto.
		\item RF-2.3 Obtener los foros de un curso: la aplicación debe ser capaz de obtener los foros, así como sus discusiones, de un curso concreto.
	\end{itemize}
	\item RF-3 Interacción por voz: la aplicación debe permitir al usuario interactuar con ella mediante comandos de voz
	\item RF-4 Interacción por texto: la aplicación debe permitir al usuario interactuar con ella mediante texto
	\item RF-5 Logs de ejecución: la aplicación debe guardar y visualizar mensajes describiendo la ejecución.
	\begin{itemize}
		\item RF-5.1 Guardar logs: la aplicación debe guardar logs de la ejecución.
		\item RF-5.1 Visualizar logs: la aplicación debe mostrar logs de la ejecución.
	\end{itemize}
	\item RF-6 Activar/desactivar \textit{skills}: la aplicación debe permitir al usuario activar o desactivar \textit{skills} en ejecución.
	\item RF-7 Control del micrófono: la aplicación debe permitir al usuario silenciar o activar el micrófono para utilizar los comandos de voz.
\end{itemize}

\subsection{Requisitos no funcionales}
\begin{itemize}
	\item RNF-1 Usabilidad: el usuario debe ser capaz de utilizar la aplicación sin mucho esfuerzo.
	\item RNF-2 Internacionalización: la aplicación debe permitir introducir nuevos idiomas y elegir entre estos.
\end{itemize}

\section{Especificación de requisitos}
	
	% FORMATO DE LAS TABLAS SACADO DE https://github.com/aog0036/TFG-SmartBeds/blob/master/doc/tex/B_Requisitos.tex
	
	\tablaSmallSinColores{Caso de uso 1:  Comando de voz }{p{3cm} p{.75cm} p{9cm}}{comando_voz}{
		\multicolumn{3}{p{10.25cm}}{CU-1: Comando de voz} \\
	}
	{
		Descripción                            & \multicolumn{2}{p{10.25cm}}{La aplicación deberá responder al usuario tras el comando de voz.} \\\hubu
		Precondiciones                         & \multicolumn{2}{p{10.25cm}}{El usuario se ha logueado en Moodle y el micrófono está activo.} \\\hubu
		\multirow{3}{3.5cm}{Secuencia}  & Paso & Acción \\\cline{2-3}
		& 1    & El usuario hace la pregunta por voz. \\\cline{2-3}
		& 2    & Mycroft convierte la pregunta a texto (STT). \\\cline{2-3}
		& 3    & Mycroft ejecuta el código correspondiente a la pregunta. \\\cline{2-3}
		& 4    & Mycroft convierte la respuesta a voz (TTS). \\\cline{2-3}
		& 5	   & El usuario obtiene la respuesta. \\\hubu
		Postcondiciones                        & \multicolumn{2}{p{10.25cm}}{} \\
	}

	\tablaSmallSinColores{Caso de uso 2:  Pregunta por texto }{p{3cm} p{.75cm} p{9cm}}{pregunta_texto}{
		\multicolumn{3}{p{10.25cm}}{CU-2: Pregunta por texto} \\
	}
	{
		Descripción                            & \multicolumn{2}{p{10.25cm}}{La aplicación deberá responder al usuario tras la pregunta por texto.} \\\hubu
		Precondiciones                         & \multicolumn{2}{p{10.25cm}}{El usuario se ha logueado en Moodle.} \\\hubu
		\multirow{3}{3.5cm}{Secuencia}  & Paso & Acción \\\cline{2-3}
		& 1    & El usuario introduce la pregunta. \\\cline{2-3}
		& 2    & La aplicación envía la pregunta a Mycroft. \\\cline{2-3}
		& 3    & Mycroft ejecuta el código correspondiente a la pregunta. \\\cline{2-3}
		& 4    & Mycroft convierte la respuesta a voz (TTS). \\\cline{2-3}
		& 5	   & El usuario obtiene la respuesta. \\\hubu
		Postcondiciones                        & \multicolumn{2}{p{10.25cm}}{} \\
	}

	\tablaSmallSinColores{Caso de uso 3:  Recordar credenciales }{p{3cm} p{.75cm} p{9cm}}{recordar_credenciales}{
		\multicolumn{3}{p{10.25cm}}{CU-3: Recordar credenciales} \\
	}
	{
		Descripción                            & \multicolumn{2}{p{10.25cm}}{La aplicación deberá recordar los credenciales de inicio de sesión.} \\\hubu
		Precondiciones                         & \multicolumn{2}{p{10.25cm}}{} \\\hubu
		\multirow{3}{3.5cm}{Secuencia}  & Paso & Acción \\\cline{2-3}
		& 1    & El usuario introduce sus credenciales. \\\cline{2-3}
		& 2    & El usuario marca las casillas de recordar. \\\cline{2-3}
		& 3    & El usuario hace click en el botón acceder. \\\cline{2-3}
		& 4    & La aplicación intenta loguear al usuario. \\\cline{2-3}
		& 5	   & Si los credenciales son correctos la aplicación guarda los credenciales. \\\hubu
		Postcondiciones                        & \multicolumn{2}{p{10.25cm}}{} \\
	}

	\tablaSmallSinColores{Caso de uso 4:  Activar o desactivar micrófono }{p{3cm} p{.75cm} p{9cm}}{cambiar_microfono}{
		\multicolumn{3}{p{10.25cm}}{CU-4: Activar o desactivar micrófono} \\
	}
	{
		Descripción                            & \multicolumn{2}{p{10.25cm}}{La aplicación deberá cambiar el estado del micrófono.} \\\hubu
		Precondiciones                         & \multicolumn{2}{p{10.25cm}}{} \\\hubu
		\multirow{3}{3.5cm}{Secuencia}  & Paso & Acción \\\cline{2-3}
		& 1    & El usuario hace click en el botón del micrófono. \\\cline{2-3}
		& 2    & La aplicación cambia el estado del micrófono. \\\hubu
		Postcondiciones                        & \multicolumn{2}{p{10.25cm}}{} \\
	}

	\tablaSmallSinColores{Caso de uso 5:  Activar o desactivar \textit{skill} }{p{3cm} p{.75cm} p{9cm}}{cambiar_skill}{
		\multicolumn{3}{p{10.25cm}}{CU-5: Activar o desactivar \textit{skill}} \\
	}
	{
		Descripción                            & \multicolumn{2}{p{10.25cm}}{La aplicación deberá cambiar el estado de las \textit{skills}.} \\\hubu
		Precondiciones                         & \multicolumn{2}{p{10.25cm}}{} \\\hubu
		\multirow{3}{3.5cm}{Secuencia}  & Paso & Acción \\\cline{2-3}
		& 1    & El usuario hace click en el botón de administrar \textit{skills}. \\\cline{2-3}
		& 2    & El usuario marca las casillas de las \textit{skills} que quiere que estén activas. \\\cline{2-3}
		& 3    & El usuario hace click en el botón guardar. \\\cline{2-3}
		& 4    & La aplicación guarda el estado de las \textit{skills}. \\\hubu
		Postcondiciones                        & \multicolumn{2}{p{10.25cm}}{} \\
	}

	\tablaSmallSinColores{Caso de uso 6:  Ver logs \textit{skill} }{p{3cm} p{.75cm} p{9cm}}{ver_logs}{
		\multicolumn{3}{p{10.25cm}}{CU-6: Ver logs \textit{skill}} \\
	}
	{
		Descripción                            & \multicolumn{2}{p{10.25cm}}{La aplicación deberá mostrar los logs.} \\\hubu
		Precondiciones                         & \multicolumn{2}{p{10.25cm}}{El usuario se ha logueado en Moodle.} \\\hubu
		\multirow{3}{3.5cm}{Secuencia}  & Paso & Acción \\\cline{2-3}
		& 1    & El usuario hace click en el botón de logs. \\\cline{2-3}
		& 2    & La aplicación limpia los logs si es necesario. \\\cline{2-3}
		& 3    & La aplicación muestra los logs. \\\cline{2-3}
		& 4    & El usuario lee los logs. \\\hubu
		Postcondiciones                        & \multicolumn{2}{p{10.25cm}}{} \\
	}

\imagen{diagrama_casos_uso}{Diagrama de casos de uso}

