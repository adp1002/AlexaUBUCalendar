\capitulo{7}{Conclusiones y Líneas de trabajo futuras}

\section{Conclusiones}

Una de las conclusiones que he sacado de este proyecto es que, aunque te encuentres con grandes problemas como fue el prototipo de skill para Alexa, en este tipo de proyectos es importante no abandonar y buscar otro camino, porque aunque parezca una pérdida de tiempo o algo problemático, puede ser que llegues a un mejor resultado que el esperado, como por ejemplo poder hacer una interfaz gráfica tras el cambio a Mycroft.

Otra conclusión a la que se ha llegado es que es importante dedicarle más tiempo a pensar qué es lo que quieres hacer que ponerse a hacer sin tener un plan, porque al final acabas dedicando más tiempo arreglando y cambiando cosas que el que dedicas cuando piensas las cosas.

Como conclusión final, se puede afirmar que se han cumplido los objetivos del proyecto. El producto final es una aplicación que permite al usuario interactuar con una plataforma de Moodle mediante voz y texto, configurar el asistente de voz y multitud más de funcionalidad.

Se han aplicado muchos de los conocimientos aprendidos durante el grado, como puede ser la interacción hombre-máquina, buenas prácticas de programación, defectos de diseño, refactorización de código, etc.

También se ha aprendido mucho en cuanto a la realización de interfaces gráficas, con las que no tenía apenas experiencia.

En definitiva, gracias a lo aprendido en la carrera se ha creado una aplicación funcional, útil y cumpliendo con los objetivos del proyecto. Ha sido una experiencia bastante agradable, aunque a veces te encuentres con problemas es muy satisfactorio solventarlos y te motiva para continuar con el proyecto y explorar un campo completamente desconocido para mí, como eran los asistentes de voz, es muy entretenido y se aprende mucho.

\section{Líneas de trabajo futuras}

Durante la realización del proyecto han surgido varias ideas sobre cómo se puede mejorar la aplicación.

Una de las mayores desventajas que tuvo cambiar de Alexa a Mycroft es que Mycroft no tiene soporte para Windows. Sin embargo, es posible ejecutar Mycroft desde Docker\cite{DockerSoftware2020}, que permite automatizar el despliegue de aplicaciones, por lo que se podría ejecutar la aplicación con Docker y utilizarla conectándose al contenedor desde Windows. Además también es posible utilizar Mycroft desde Android, que facilitaría mucho su uso para los estudiantes.

Otra posible continuación del desarrollo es guardar los datos que se obtienen de Moodle en ficheros locales de forma que se pueda utilizar la aplicación sin conexión a Internet. Usar la aplicación \textit{offline} reduciría bastante la funcionalidad de la aplicación, principalmente el servicio de STT, aunque gracias a que Mycroft es modular se pueden cambiar estos elementos. De hecho, el motor de TTS por defecto en Mycroft, Mimic\cite{Mimic} permite instalarse y utilizarse de forma local. En este \href{https://community.mycroft.ai/t/can-i-use-mycroft-offline/5306/5}{post del foro} de Mycroft se detalla bastante qué cambios son necesarios para utilizar Mycroft sin conexión.

Una de las formas más claras de mejorar la aplicación es aumentar la funcionalidad de la aplicación, permitiendo al usuario obtener más información de Moodle.

Por último se puede mejorar la interacción con el usuario. Esto implica mejorar la interfaz gráfica, que puede ser añadiendo funcionalidad o mejorar la claridad visual, y mejorar la interacción con Mycroft, de forma que las conversaciones sean más "humanas".
